%%
%% This is file `sample-sigconf.tex',
%% generated with the docstrip utility.
%%
%% The original source files were:
%%
%% samples.dtx  (with options: `sigconf')
%% 
%% IMPORTANT NOTICE:
%% 
%% For the copyright see the source file.
%% 
%% Any modified versions of this file must be renamed
%% with new filenames distinct from sample-sigconf.tex.
%% 
%% For distribution of the original source see the terms
%% for copying and modification in the file samples.dtx.
%% 
%% This generated file may be distributed as long as the
%% original source files, as listed above, are part of the
%% same distribution. (The sources need not necessarily be
%% in the same archive or directory.)
%%
%%
%% Commands for TeXCount
%TC:macro \cite [option:text,text]
%TC:macro \citep [option:text,text]
%TC:macro \citet [option:text,text]
%TC:envir table 0 1
%TC:envir table* 0 1
%TC:envir tabular [ignore] word
%TC:envir displaymath 0 word
%TC:envir math 0 word
%TC:envir comment 0 0
%%
%%
%% The first command in your LaTeX source must be the \documentclass
%% command.
%%
%% For submission and review of your manuscript please change the
%% command to \documentclass[manuscript, screen, review]{acmart}.
%%
%% When submitting camera ready or to TAPS, please change the command
%% to \documentclass[sigconf]{acmart} or whichever template is required
%% for your publication.
%%
%%
\documentclass[sigconf]{acmart}

%%
%% Added by Ed:
\usepackage{eucal}

%%
%% \BibTeX command to typeset BibTeX logo in the docs
\AtBeginDocument{%
  \providecommand\BibTeX{{%
    Bib\TeX}}}

%% Rights management information.  This information is sent to you
%% when you complete the rights form.  These commands have SAMPLE
%% values in them; it is your responsibility as an author to replace
%% the commands and values with those provided to you when you
%% complete the rights form.
\setcopyright{acmcopyright}
\copyrightyear{2023}
\acmYear{2023}
\acmDOI{XXXXXXX.XXXXXXX}

%% These commands are for a PROCEEDINGS abstract or paper.
\acmConference[Conference acronym 'XX]{Make sure to enter the correct
  conference title from your rights confirmation email}{June 03--05,
  2018}{Woodstock, NY}
%%
%%  Uncomment \acmBooktitle if the title of the proceedings is different
%%  from ``Proceedings of ...''!
%%
%%\acmBooktitle{Woodstock '18: ACM Symposium on Neural Gaze Detection,
%%  June 03--05, 2018, Woodstock, NY}
\acmPrice{15.00}
\acmISBN{978-1-4503-XXXX-X/18/06}


%%
%% Submission ID.
%% Use this when submitting an article to a sponsored event. You'll
%% receive a unique submission ID from the organizers
%% of the event, and this ID should be used as the parameter to this command.
%%\acmSubmissionID{123-A56-BU3}

%%
%% For managing citations, it is recommended to use bibliography
%% files in BibTeX format.
%%
%% You can then either use BibTeX with the ACM-Reference-Format style,
%% or BibLaTeX with the acmnumeric or acmauthoryear sytles, that include
%% support for advanced citation of software artefact from the
%% biblatex-software package, also separately available on CTAN.
%%
%% Look at the sample-*-biblatex.tex files for templates showcasing
%% the biblatex styles.
%%

%%
%% The majority of ACM publications use numbered citations and
%% references.  The command \citestyle{authoryear} switches to the
%% "author year" style.
%%
%% If you are preparing content for an event
%% sponsored by ACM SIGGRAPH, you must use the "author year" style of
%% citations and references.
%% Uncommenting
%% the next command will enable that style.
%%\citestyle{acmauthoryear}


%%
%% end of the preamble, start of the body of the document source.
\begin{document}

%%
%% The "title" command has an optional parameter,
%% allowing the author to define a "short title" to be used in page headers.
\title{Application of Invariant Feature-Based Multi-Reference Alignment on Lead-Lag Detection}

%%
%% The "author" command and its associated commands are used to define
%% the authors and their affiliations.
%% Of note is the shared affiliation of the first two authors, and the
%% "authornote" and "authornotemark" commands
%% used to denote shared contribution to the research.
\author{Danni Shi}
\email{danni.shi@eng.ox.ac.uk}
% \orcid{1234-5678-9012}
\affiliation{%
  \institution{University of Oxford}
  \city{Oxford}
  \country{UK}
}

\author{Mihai Cucuringu}
\email{mihai.cucuringu@stats.ox.ac.uk}
\affiliation{%
  \institution{Department of Statistics, University of Oxford}
  \city{Oxford}
  \country{UK}}
\affiliation{%
  \institution{The Alan Turing Institute}
  \city{London}
  \country{UK}}

\author{Jan-Peter Calliess}
\email{jan-peter.calliess@oxford-man.ox.ac.uk}
\affiliation{%
  \institution{University of Oxford}
  \city{Oxford}
  \country{UK}
}


%% The abstract is a short summary of the work to be presented in the
%% article.
\begin{abstract}
 In time series analysis, numerous methods have been developed to detect, measure and apprehend lead-lag relationships between variables, which refers to the time-delay of patterns in a time series relative to the other. Often, the extent of lead-lag between two time series is directly measured with a similarity metric. This approach suffers inconsistency and inaccuracy especially under high-noise regimes. 

This work represents several frameworks of Multireference alignment (MRA), the estimation of a unified, de-noised latent signal from a population of time series with linearly-shifted and noisy patterns, and their application on lead-lag detection. The recovered latent signal is treated as the reference vector whose lead-lag metric is calculated against 
other time series. The relative lead-lag between two actual time series is then obtained as the difference between their lags against the reference vector.  Results show that models with MRA-induced intermediates outperform the simplistic models relying on pairwise similarity in predicting the relative shifts between time series when the signal-to-noise ratio (SNR) is low. The effect of clustering the time series advancing the recovery of the latent signals is also investigated

The results of lead-lag detection can be used on trading stock returns. We developed a trading framework that utilises the relative shift estimations to construct financial signals from the leading time series and trade on multiple portfolios of the lagging time series. 
\end{abstract}

%%
%% The code below is generated by the tool at http://dl.acm.org/ccs.cfm.
%% Please copy and paste the code instead of the example below.
%%
\begin{CCSXML}
<ccs2012>
 <concept>
  <concept_id>10010520.10010553.10010562</concept_id>
  <concept_desc>Computer systems organization~Embedded systems</concept_desc>
  <concept_significance>500</concept_significance>
 </concept>
 <concept>
  <concept_id>10010520.10010575.10010755</concept_id>
  <concept_desc>Computer systems organization~Redundancy</concept_desc>
  <concept_significance>300</concept_significance>
 </concept>
 <concept>
  <concept_id>10010520.10010553.10010554</concept_id>
  <concept_desc>Computer systems organization~Robotics</concept_desc>
  <concept_significance>100</concept_significance>
 </concept>
 <concept>
  <concept_id>10003033.10003083.10003095</concept_id>
  <concept_desc>Networks~Network reliability</concept_desc>
  <concept_significance>100</concept_significance>
 </concept>
</ccs2012>
\end{CCSXML}

\ccsdesc[500]{Computer systems organization~Embedded systems}
\ccsdesc[300]{Computer systems organization~Redundancy}
\ccsdesc{Computer systems organization~Robotics}
\ccsdesc[100]{Networks~Network reliability}

%%
%% Keywords. The author(s) should pick words that accurately describe
%% the work being presented. Separate the keywords with commas.
\keywords{High-dimensional time series, lead-lag, clustering, financial markets}
%% A "teaser" image appears between the author and affiliation
%% information and the body of the document, and typically spans the
%% page.

\received{20 February 2007}
\received[revised]{12 March 2009}
\received[accepted]{5 June 2009}

%%
%% This command processes the author and affiliation and title
%% information and builds the first part of the formatted document.
\maketitle



\section{Introduction}

\subsection{Background}
Time series forecasting is a major topic in data science and has been extensively researched from multiple perspectives. Forecasting financial time series is especially challenging due to low SNR and the deficiency of auto-correlation. Yet, in the financial market, it has been observed that the prices of certain groups of assets respond to events and factors quicker than others. In multivariate time series models, market factors can create similar patterns of change in the financial time series of different assets, each with a different time delay. This is what we call a lead-lag structure amongst the time series variables. We aim to construct a measure that estimates the extent and length of lead-lag between two stocks. In this paper, the extent of lead-lag is represented as the maximum cross-correlation between two time series up to a shift. The corresponding length of lead-lag is the number of timescale shifts (as we primarily works with discrete-time data in finance) to achieve the maximum cross-correlation. 

Current methods of lead-lag detection often rely on computing a lead-lag score between a pair of time series (some refs and descriptions here), which are direct and effective in some use cases. However, several problems arise with this approach:
\begin{itemize}
\item Non-robust: At low SNR, pairwise measurement is significantly interfered by the noise in the data and hence become inaccurate
\item Inconsistent: Pairwise lag predictions need to be synchronized for reliable applications. However, the lag between a pair of time series may not agree with the others. \footnote{For example, estimations may show that variable $X$ and $Y$ each leads variable $Z$ by 1 day, but that $X$ lags $Y$ by 1 day too.}
\item Computationally expensive: The computation of the pairwise metric of $N$ time series has a complexity $\mathcal{O}(N^2)$, making it non-scalable to large datasets.
\end{itemize}

Some algorithms aim to address the inconsistencies from pairwise measurements by forming a unifying vector of ranking and/or shifts (refs here). The recovered vector can be seen as a low-dimensional representation that closely summarise the original pairwise shift matrix. An SVD-based algorithm is shown to recover the ranking of a group of pairwise measurements with efficiency and robustness.

MRA is studied in multiple scientific domains such as structural biology, radar and image processing where observations are abundant yet noisy and cyclically shifted. Some methods tap on the signal features invariant under cyclic shifts and the Central Limit Theorem to construct an optimization problem\cite{Boumal2017Oct}. The problem converges to an estimated latent signal and was shown to perform well at high noise levels as long as a sufficient number of observations are available. 

\subsection{Our contributions}
    
In this paper, we propose a framework to estimate the relative shifts of multiple time series. We model financial time series in the same universe as the noisy and linearly shifted copies of several latent source signals. Clustering is performed to identify the time series coming from the same source signal. After that, we investigate the accuracy and limitations of 3 different MRA methods in terms of signal recovery and lead-lag detection. This part is implemented on synthetic data where the ground truths of latent signal and shifts are known. Furthermore, we separate the time series into groups of the same shift and devise a trading strategy that derive financial signals from the leading groups and trade on a basket of lagging stocks. To demonstrate the application in financial markets, we implement the clustering $\rightarrow$ lead-lag detection $\rightarrow$ trading pipeline on a dataset of 695 US equities' returns. With comparison to the baseline pairwisely evaluated model, the 3 MRA-based methods exhibit strong ability to estimate lead-lag relationships at high noise regimes and produce more significant portfolio returns from the trading strategy.

\section{Lead-Lag Detection Based on Cross-Correlation}
\subsection{Pairwise Lag Measurement}
\subsection{Relative Lag Measurement with a Reference Signal}

\section{Multireference Alignment}
\subsection{Homogeneous MRA}
The original method aims to recover the original signal from cyclically shifted copies of noisy observations while the financial data we work with contains non-cyclic shifts. We make an educated assumption that the shifts are small compared to the length of the time series so that a large proportion of the time series agrees with the cyclically shifted version. 
\subsection{Heterogeneous MRA}
\subsection{MRA Based on Correlation and Synchronization}

\section{Experiments and Results}
\subsection{Synthetic Data}
\subsection{Financial Data}

\section{Discussions}
\section{Conclusions}

\section{References}



%%
%% The acknowledgments section is defined using the "acks" environment
%% (and NOT an unnumbered section). This ensures the proper
%% identification of the section in the article metadata, and the
%% consistent spelling of the heading.
\begin{acks}
To Robert, for the bagels and explaining CMYK and color spaces.
\end{acks}

%%
%% The next two lines define the bibliography style to be used, and
%% the bibliography file.
\bibliographystyle{ACM-Reference-Format}
\bibliography{bibliography}


\end{document}
\endinput
%%
%% End of file `sample-sigconf.tex'.